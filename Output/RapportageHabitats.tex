\documentclass[twoside]{extreport}

\usepackage{inborapport_2015}
\codesize{\footnotesize}




\title{Analyse van de gegevens van het meetnet habitatkwaliteit ten behoeve van
de rapportage voor de Habitatrichtlijn (periode 2013-2018)}
\subtitle{Technisch rapport}
\author{Toon Westra, An Leyssen, Patrik Oosterlynck, Els Lommelen, Jeroen Vanden
Borre, Steven De Saeger, Bart Vandevoorde, Arno Thomaes, Sam Provoost,
Desiré Paelinckx}

\reportnumber{RapportNr}






% Alter some LaTeX defaults for better treatment of figures:
% See p.105 of "TeX Unbound" for suggested values.
% See pp. 199-200 of Lamport's "LaTeX" book for details.
%   General parameters, for ALL pages:
\renewcommand{\topfraction}{0.9}	% max fraction of floats at top
\renewcommand{\bottomfraction}{0.8}	% max fraction of floats at bottom
%   Parameters for TEXT pages (not float pages):
\setcounter{topnumber}{2}
\setcounter{bottomnumber}{2}
\setcounter{totalnumber}{4}     % 2 may work better
\setcounter{dbltopnumber}{2}    % for 2-column pages
\renewcommand{\dbltopfraction}{0.9}	% fit big float above 2-col. text
\renewcommand{\textfraction}{0.07}	% allow minimal text w. figs
%   Parameters for FLOAT pages (not text pages):
\renewcommand{\floatpagefraction}{0.7}	% require fuller float pages
% N.B.: floatpagefraction MUST be less than topfraction !!
\renewcommand{\dblfloatpagefraction}{0.7}	% require fuller float pages

\begin{document}
\maketitle
\pagenumbering{arabic}


%starttoc

\clearpage

\phantomsection
\setcounter{tocdepth}{3}
\tableofcontents
\addcontentsline{toc}{chapter}{\contentsname}

\clearpage

\phantomsection
\listoffigures
\addcontentsline{toc}{chapter}{\listfigurename}
\vspace{34pt}

\phantomsection
\listoftables
\addcontentsline{toc}{chapter}{\listtablename}

\clearpage

%endtoc

\chapter*{Samenvatting}\label{samenvatting}
\addcontentsline{toc}{chapter}{Samenvatting}

\chapter*{English abstract}\label{english-abstract}
\addcontentsline{toc}{chapter}{English abstract}

\benglish

\eenglish

\chapter{Inleiding}\label{inleiding}

\section{Situering}\label{situering}

Om de zes jaar rapporteert het Instituut voor Natuur- en Bosonderzoek
(INBO) over de staat van instandhouding van de habitattypen van de
Habitatrichtlijn. De meest recente rapportage is terug te vinden in
\citet{Paelinckx2019}. In deze rapportage wordt voor het eerst gebruik
gemaakt van gegevens van het meetnet habitatkwaliteit \citep{Westra2014}
dat INBO en het Agentschap voor Natuur en Bos (ANB) opstartten in 2014.

De staat van instandhouding van de habitattypen is gebaseerd op volgende
onderdelen:

\begin{itemize}
\tightlist
\item
  verspreiding,
\item
  oppervlakte,
\item
  specifieke structuren en functies (incl. habitattypische soorten),
\item
  toekomstperspectieven.
\end{itemize}

De gegevens van het meetnet habitatkwaliteit worden gebruikt om het
onderdeel specifieke structuren en functies te beoordelen. In dit
rapport geven we de technische achtergrond van de analyse van deze
gegevens in functie van de rapportage van \citet{Paelinckx2019}.

\section{Meetnet habitatkwaliteit}\label{meetnet-habitatkwaliteit}

\subsection{Meetnetontwerp}\label{h:meetnetontwerp}

Het meetnet habitatkwaliteit bestaat uit een steekproef van meetpunten
voor elk habitattype in Vlaanderen, uitgezonderd de zeer zeldzame
habitattypen \citep{Westra2014}. De zeer zeldzame habitattypen worden
gebiedsdekkend opgevolgd via een kartering.

Voor de terrestrische habitattypen bestaat een meetpunt uit een
vierkante plot (3 m x 3 m voor open habitattypen en 9 m x 9 m voor
boshabitattypen) waarin een vegetatieopname \citep{INBO2016a} wordt
gemaakt, en een cirkelplot (straal 18 m) waarin bijkomende variabelen
worden bepaald die voornamelijk betrekking hebben op de
habitatstructuur.

Voor aquatische habitattypen in stilstaande wateren bestaat een meetpunt
uit een volledig waterlichaam. Voor aquatische habitattypen in stromende
wateren bestaat een meetpunt uit een riviersegment van 100 meter, waarin
gegevens worden ingezameld overeenkomstig het veldprotocol van
\citet{INBO2017}.

Op basis van de ingezamelde gegevens kunnen voor elk meetpunt de
indicatoren van de Lokale Staat van Instandhouding, zowel LSVI versie 2
\citep{TJollyn2009} als LSVI versie 3 \citep{Oosterlynck2018}, berekend
worden. Vervolgens kan op basis hiervan een schatting gemaakt worden van
het oppervlakteaandeel van een habitattype in Vlaanderen dat een
gunstige habitatkwaliteit heeft.

De meetpunten van het meetnet habitatkwaliteit worden bemonsterd over
een periode van 12 jaar (= de meetcyclus), waarbij er getracht wordt elk
jaar een random gekozen subset van 1/12de van de meetpunten af te
werken. Gezien het meetnet pas in 2014 werd opgestart, is er meestal nog
maar een beperkt aandeel van de volledige steekproefgrootte afgerond.
Toch kunnen we op basis hiervan al representatieve schattingen maken van
het aandeel habitat met gunstige habitatkwaliteit. De precisie zal
uiteraard lager zijn, wat zich vertaalt in bredere
betrouwbaarheidsintervallen.

Uit de analyse van de informatiebehoefte \citep{Adriaens2011} bleek dat
men een grotere precisie wenste te bekomen binnen
Habitatrichtlijngebieden (SBZ-H) en dat men ook een uitspraak over de
habitatkwaliteit per habitatsubtype wenste te bekomen. Daarom werd er
een relatief groter aantal meetpunten binnen SBZ-H geselecteerd en
werden de relatief minder algemene habitatsubtypen overbemonsterd.
Wanneer we een representatieve uitspraak willen per habitattype voor
Vlaanderen (zowel binnen als buiten SBZ-H), moeten we hiermee rekening
houden. Dit kan door gebruik te maken van meetpuntgewichten (zie
paragraaf \ref{h:meetpuntgewichten}).

\subsection{Synergie andere
meetnetten}\label{synergie-andere-meetnetten}

Voor sommige habitattypen lopen er al langer monitoringprogramma's op
basis waarvan de habitatkwaliteit kan ingeschat worden. We maken
uiteraard zo veel mogelijk gebruik van de gegevens van deze
monitoringprogramma's. Het gaat om de volgende monitoringprogramma's:

\begin{itemize}
\tightlist
\item
  Vlaamse Bosinventarisatie (VBI) voor boshabitattypen
  \citep{Wouters2008c};
\item
  Permanente Inventarisatie van de Natuurreservaten aan de Kust (PINK)
  voor kustduinhabitattypen \citep{Provoost2015};
\item
  Geïntegreerde systeemmonitoring van het Schelde-estuarium (MONEOS)
  voor wilgenvloedbossen (91E0\_sf) en buitendijkse schorren (1330\_da)
  \citep{VanRyckegem2018}.
\end{itemize}

\section{Opbouw rapport}\label{opbouw-rapport}

Eerst overlopen we de generieke stappen van de analyse, gevolgd door een
detailbespreking per dataset. Deze datasets komen niet volledig overeen
met de habitattypegroepen, wat een gevolg is van de manier waarop de
gegevensinzameling georganiseerd is. De gegevensinzameling gebeurt door
verschillende teams van zowel het INBO als het ANB, waardoor er hier en
daar wat verschillen zijn in de manier waarop de data worden opgeslagen.
Voor de analyse is het dan ook eenvoudiger om deze dataset per dataset
uit te voeren.

\chapter{Analysestappen}\label{analysestappen}

Voor de analyse maken we gebruik van de LSVI-rekenmodule. Dit is een
R-package genaamd `LSVI' ontwikkeld door \citet{Lommelen2019}. De
LSVI-rekenmodule maakt gebruik van een databank met alle kenmerken van
de LSVI-indicatoren, zoals soortenlijsten en drempelwaarden voor een
gunstige staat. Op basis van deze databank en op basis van
terreingegevens berekent de LSVI-rekenmodule de waarden voor de
verschillende indicatoren van een bepaald habitat(sub)type. Daarnaast
wordt er ook een beoordeling gemaakt per indicator en een geïntegreerde
beoordeling over alle indicatoren heen.

De R-code van de analyses voor dit rapport kan geraadpleegd worden op
\href{https://github.com/ToonHub/RapportageHabitats/}{Github}.

\section{Inlezen ruwe data}\label{inlezen-ruwe-data}

In een eerste stap worden de ruwe data ingelezen. De meeste gegevens die
door INBO worden ingezameld zitten in de
\href{https://inboveg.inbo.be/}{Vlaamse databank vegetatieopnamen
(INBOVEG)}. ANB maakt gebruik van FieldMap voor de invoer van
terreingegevens. Deze gegevens zitten nog niet in INBOVEG of een andere
gecentraliseerde databank. We maken daarom gebruik van een export uit
FieldMap onder de vorm van een Acces-bestand.

\section{Ruwe data omzetten naar invoerformaat voor
LSVI-rekenmodule}\label{h:datainvoer}

De LSVI-rekenmodule vereist een specifiek invoerformaat. Onderstaande
onderdelen moeten ingevoerd worden in de rekenmodule.

In eerste instantie moet voor elk meetpunt het habitattype of het
habitatsubtype (als het habitattype is opgesplitst in verschillende
subtypen) opgegeven worden. Dit is noodzakelijk omdat de LSVI-berekening
specifiek is voor elk habitat(sub)type. De informatie m.b.t.
habitat(sub)typen wordt ingevoerd via het attribuut
\textbf{Data\_habitat}. Daarnaast worden ook de terreingegevens
ingevoerd. Een LSVI-indicator bestaat uit één of meerdere voorwaarden.
De indicator `sleutelsoorten' kan bijvoorbeeld onderverdeeld worden in
de voorwaarde `aantal sleutelsoorten' en de voorwaarde `bedekking
sleutelsoorten'. De gebruiker heeft enerzijds de mogelijkheid om de
waarden van de voorwaarden rechtstreeks in te voeren in de
LSVI-rekenmodule (bv. aantal sleutelsoorten = 3 en bedekking
sleutelsoorten = 10 \%). Deze gegevens worden ingevoerd via het
attribuut \textbf{Data\_voorwaarden}. Anderzijds kan ook een lijst met
de bedekkingen van de aanwezige soorten ingevoerd worden, op basis
waarvan de rekenmodule de waarden berekent voor de voorwaarden. Deze
gegevens worden ingevoerd via het attribuut
\textbf{Data\_soortenKenmerken}. Ook andere variabelen op basis waarvan
de waarde van voorwaarden berekend kan worden, worden ingegeven via
Data\_soortenKenmerken. Voorbeelden hiervan zijn de aanwezige
groeiklassen bij boshabitattypen (voor de berekening van de voorwaarde
`aantal groeiklassen') of het grondvlak van de aanwezige boomsoorten
(voor de berekening van de voorwaarde `grondvlakaandeel
sleutelsoorten').

Voor elke analyse bewaren we de gegevens die als invoer voor de
LSVI-rekenmodule gebruikt worden. We maken voor elke analyse een map
`InputRekenmodule' aan met volgende bestanden:

\begin{itemize}
\tightlist
\item
  data\_habitat\_\emph{naamDataset}.csv,
\item
  data\_soortenKenmerken\_\emph{naamDataset}.csv,
\item
  data\_voorwaarden\_\emph{naamDataset}.csv.
\end{itemize}

\section{Bepalen van meetpuntgewichten}\label{h:meetpuntgewichten}

Zoals vermeld in paragraaf \ref{h:meetnetontwerp} is het meetnet
habitatkwaliteit zo ontworpen dat de dichtheid aan meetpunten voor een
bepaald habitattype niet overal gelijk is. Meer bepaald zijn er
verschillen binnen en buiten SBZ-H (relatief groter aantal meetpunten
binnen SBZ-H) en tussen de verschillende habitatsubtypen (relatief
groter aantal meetpunten in de minder voorkomende habitatsubtypen). We
moeten hiermee rekening houden als we een onvertekende schatting wensen
te bekomen voor Vlaanderen. Daarnaast bestaan sommige meetpunten maar
gedeeltelijk uit het doelhabitattype. Deze meetpunten vertegenwoordigen
dus in mindere mate het doelhabitattype dan de meetpunten die volledig
uit doelhabitat bestaan.

Voor de terrestrische habitattypen onderscheiden we twee typen
gewichten:

\begin{itemize}
\tightlist
\item
  \textbf{Plotgewichten}. Dit is de fractie van de cirkelplot (straal 18
  m) die uit doelhabitat bestaat.
\item
  \textbf{Stratumgewichten}. In het meetnet habitatkwaliteit kunnen
  verschillende strata onderscheiden worden. Voor een habitattype dat
  niet is onderverdeeld in subtypen zijn de strata het deel van het
  habitattype dat gelegen is binnen SBZ-H en het deel buiten SBZ-H. Voor
  een habitattype dat wel is onderverdeeld in subtypen worden de strata
  gevormd door de combinatie van de subtypen en de ligging t.o.v. SBZ-H.
  De strata zijn in dat geval als volgt: subtype 1 binnen SBZ-H, subtype
  1 buiten SBZ-H, subtype 2 binnen SBZ-H, subtype 2 buiten SBZ-H,
  etc\ldots{} Voor elk stratum berekenen we de dichtheid van de
  meetpunten, m.a.w. het aantal (bemonsterde) meetpunten gedeeld door de
  oppervlakte van het stratum. Om een onvertekende schatting te bekomen
  voor Vlaanderen moeten we compenseren voor verschillen in de dichtheid
  van meetpunten tussen de strata. Meer bepaald moeten meetpunten in een
  stratum met een hogere dichtheid een lager gewicht krijgen dan de
  meetpunten in een stratum met een lagere dichtheid. Dit kan door een
  stratumgewicht te gebruiken dat omgekeerd evenredig is met de
  dichtheid van de meetpunten in het stratum. Dit komt dus neer op de
  oppervlakte van het stratum gedeeld door het aantal meetpunten in het
  stratum. Het stratumgewicht kan dus geïnterpreteerd worden als de
  oppervlakte doelhabitat die elk meetpunt vertegenwoordigt.
\end{itemize}

Op basis van het plotgewicht en het stratumgewicht bepalen we dan de
meetpuntgewichten, namelijk het product van beide gewichten.

De bepaling van de meetpuntgewichten voor stilstaande wateren en de
stromende wateren bespreken we in respectievelijk Hoofdstuk
\ref{h:plassen} en Hoofdstuk \ref{h:rivieren}.

\section{Berekening LSVI per
meetpunt}\label{berekening-lsvi-per-meetpunt}

We maken gebruik van de LSVI-rekenmodule om per meetpunt de volgende
zaken te berekenen:

\begin{itemize}
\tightlist
\item
  de waarde en beoordeling (gunstig/ongunstig) van de
  \textbf{voorwaarden} ;
\item
  de beoordeling (gunstig/ongunstig) van de \textbf{indicatoren};
\item
  de beoordeling (gunstig/ongunstig) van het \textbf{meetpunt/de
  habitatvlek}.
\end{itemize}

Tabel \ref{tab:tabelVoorwaarden} geeft als voorbeeld de uitkomst van de
rekenmodule voor de voorwaarden van LSVI versie 3. De tabel bevat de
uitkomst voor twee meetpunten met habitattype 4010 die geïdentificeerd
worden op basis van de kolom `ID'.

\begin{landscape}\begin{table}[!h]

\caption{\label{tab:tabelVoorwaarden}Voorbeeld van de output van de LSVI-rekenmodule voor de voorwaarden}
\centering
\fontsize{9}{11}\selectfont
\begin{tabular}[t]{cclclcccccc}
\toprule
ID & Habitattype & Indicator & Belang & Voorwaarde & Waarde & Operator & Referentiewaarde & Status\_voorwaarde & VoorwaardeID & Combinatie\\
\midrule
11218 & 4010 & dwergstruiken & b & bedekking dwergstruiken & 87.5 & >= & B & TRUE & 704 & 704\\
11218 & 4010 & sleutelsoorten & b & aantal sleutelsoorten & 1.0 & >= & 4 & FALSE & 329 & 329 AND 547\\
11218 & 4010 & sleutelsoorten & b & aantal veenmossen & 0.0 & >= & 1 & FALSE & 547 & 329 AND 547\\
11218 & 4010 & verbossing & b & bedekking verbossing & 0.0 & <= & 10 & TRUE & 1443 & 1443\\
11218 & 4010 & vergrassing & zb & bedekking vergrassing & 97.0 & <= & 50 & FALSE & 1508 & 1508\\
\addlinespace
100050 & 4010 & dwergstruiken & b & bedekking dwergstruiken & 87.5 & >= & B & TRUE & 704 & 704\\
100050 & 4010 & sleutelsoorten & b & aantal sleutelsoorten & 3.0 & >= & 4 & FALSE & 329 & 329 AND 547\\
100050 & 4010 & sleutelsoorten & b & aantal veenmossen & 0.0 & >= & 1 & FALSE & 547 & 329 AND 547\\
100050 & 4010 & verbossing & b & bedekking verbossing & 1.0 & <= & 10 & TRUE & 1443 & 1443\\
100050 & 4010 & vergrassing & zb & bedekking vergrassing & 1.0 & <= & 50 & TRUE & 1508 & 1508\\
\bottomrule
\end{tabular}
\end{table}
\end{landscape}

Voor elk meetpunt worden vier indicatoren berekend, waarvan de indicator
sleutelsoorten uit twee voorwaarden bestaat en de overige indicatoren
uit één voorwaarde. Verder toont de tabel de waarde van de voorwaarde.
In dit voorbeeld werd de waarde voor de voorwaarden `bedekking
dwergstruiken' en `bedekking verossing' rechtstreeks ingevoerd via het
attribuut Data\_voorwaarden (zie paragraaf \ref{h:datainvoer}). De
waarde voor de overige voorwaarden berekende de rekenmodule op basis van
de soortgegevens die werden ingevoerd via het attribuut
Data\_soortenKenmerken (zie paragraaf \ref{h:datainvoer}). De tabel
toont ook de operator en referentiewaarde op basis waarvan de
beoordeling gebeurt. De referentiewaarde kan een numerieke waarde zijn
of een bedekkingsklasse. In dit voorbeeld komt de referentiewaarde `B'
van de voorwaarde `bedekking dwergstruiken' overeen met de
bedekkingsklasse `bedekkend' van de beheermonitoringsschaal
\citep{INBO2017b}, wat overeenkomt met een indicatieve bedekking tussen
5\% en 25\%. De beoordeling wordt uitgedrukt in TRUE (gunstig) of FALSE
(ongunstig). Ten slotte geven de kolommen VoorwaardeID en Combinatie aan
hoe de voorwaarden gecombineerd moeten worden om tot een beoordeling van
de indicatoren te komen. Dit is enkel relevant als een indicator
meerdere voorwaarden bevat. In dit voorbeeld staat er bij sleutelsoorten
`AND' in de kolom Combinatie, wat betekent dat beide voorwaarden gunstig
moeten zijn voor een gunstige beoordeling van de indicator (Tabel
\ref{tab:tabelIndicatoren}). Voorwaarden kunnen ook gecombineerd worden
via `OR', wat dan betekent dat slechts één van de voorwaarden gunstig
moet zijn voor een gunstige beoordeling van de indicator.

\begin{table}[!h]

\caption{\label{tab:tabelIndicatoren}Voorbeeld van de output van de LSVI-rekenmodule voor de indicatoren}
\centering
\fontsize{9}{11}\selectfont
\begin{tabular}[t]{cclcc}
\toprule
ID & Habitattype & Indicator & Belang & Status\_indicator\\
\midrule
11218 & 4010 & dwergstruiken & b & TRUE\\
11218 & 4010 & sleutelsoorten & b & FALSE\\
11218 & 4010 & verbossing & b & TRUE\\
11218 & 4010 & vergrassing & zb & FALSE\\
100050 & 4010 & dwergstruiken & b & TRUE\\
\addlinespace
100050 & 4010 & sleutelsoorten & b & FALSE\\
100050 & 4010 & verbossing & b & TRUE\\
100050 & 4010 & vergrassing & zb & TRUE\\
\bottomrule
\end{tabular}
\end{table}

De beoordeling van het meetpunt/de habitatvlek is gebaseerd op de
beoordeling van de verschillende indicatoren en het belang van de
indicatoren uitgedrukt in `belangrijk' (b) of `zeer belangrijk' (zb). De
volgende regel geldt: een meetpunt is gunstig als meer dan 50 \% van de
indicatoren gunstig zijn en geen enkele zeer belangrijke indicator
ongunstig is \citep{Paelinckx2019}. Tabel \ref{tab:tabelStatus} geeft de
beoordeling voor beide meetpunten. Het meetpunt met ID 11218 scoort
ongunstig omdat de zeer belangrijke indicator vergrassing ongunstig
scoort. Het meetpunt met ID 100050 scoort gunstig: alle zeer belangrijke
indicatoren zijn gunstig en meer dan de helft van alle indicatoren zijn
gunstig.

\begin{table}[!h]

\caption{\label{tab:tabelStatus}Voorbeeld van de output van de LSVI-rekenmodule voor de status de habitatvlek}
\centering
\fontsize{9}{11}\selectfont
\begin{tabular}[t]{ccc}
\toprule
ID & Habitattype & Status\_habitatvlek\\
\midrule
11218 & 4010 & FALSE\\
100050 & 4010 & TRUE\\
\bottomrule
\end{tabular}
\end{table}

De resultaten worden weggeschreven in drie bestanden:

\begin{itemize}
\tightlist
\item
  Voorwaarden\_\emph{naamDataset}.csv
\item
  Indicatoren\_\emph{naamDataset}.csv
\item
  StatusHabitatvlek\_\emph{naamDataset}.csv
\end{itemize}

\section{Schatting aandeel habitat met gunstige
kwaliteit}\label{schatting-aandeel-habitat-met-gunstige-kwaliteit}

Op basis van de beoordelingen per meetpunt maken we per habitattype een
schatting van het aandeel habitat dat gunstig is en berekenen we het
bijhorende 95\%-betrouwbaarheidsinterval. We doen dit zowel voor
Vlaanderen als voor het deel van Vlaanderen dat in de Atlantische
biogeografische regio gelegen is (het deel van Vlaanderen zonder
Voeren). Voor veel habitattypen is het onderscheid tussen Vlaanderen en
Vlaanderen-Atlantisch evenwel niet relevant, omdat het habitattype enkel
in de Atlantische regio voorkomt (bv. kusthabitattypen) of omdat de
gebruikte dataset nog geen bemonsterde meetpunten uit Voeren bevat.

Naast een schatting per habitattype maken we ook een schatting van het
aandeel:

\begin{itemize}
\tightlist
\item
  per habitatsubtype,
\item
  per habitattype binnen SBZ-H,
\item
  per habitattype buiten SBZ-H.
\end{itemize}

We schatten de betrouwbaarheidsintervallen op basis van een binomiaal
model, zodat deze steeds tussen de 0 en 100\% gelegen zijn. Om de
meetpuntgewichten op een correcte wijze toe te passen in de analyse,
maken we gebruik van het R-package survey \citep{Lumley2019}.

Het resultaat wordt weggeschreven in het bestand
StatusHabitat\_\emph{naamDataset}.csv.

\chapter{Graslanden, moerassen en zilte graslanden
(1330\_hpr)}\label{h:GrasMoeras}

\section{Data}\label{data}

\subsection{Ruwe data uit INBOVEG}\label{ruwe-data-uit-inboveg}

De INBOVEG-databank bevat de volgende gegevens voor elke meetpunt:

\begin{itemize}
\item
  bedekking van de aanwezige vegetatietypen in de plot:
  habitat(sub)typen, regionaal belangrijke biotopen (RBB) of andere
  klassen (in een plot kunne meerdere klassen voorkomen),
\item
  vegetatieopname (lijst van alle aanwezige soorten + bedekking per
  soort),
\item
  bedekking van vegetatielagen en structuurvariabelen die nodig zijn
  voor de LSVI-bepaling (zoals verbossing).
\end{itemize}

Er werd zowel in de vierkante plot als in de cirkelplot een
vegetatieopname uitgevoerd. Bijkomende structuurgegevens werden enkel in
de cirkelplot opgemeten.

\subsection{Geobserveerd
habitat(sub)type}\label{geobserveerd-habitatsubtype}

In principe wordt een meetpunt enkel opgemeten als het centrum van de
plot in het doelhabitat valt (het doelhabitat is het habitat(sub)type
waarvoor het meetpunt geselecteerd werd). In sommige gevallen zien we
dat een meetpunt geen doelhabitat bevat, maar dat er toch een opname is
uitgevoerd. We nemen het meetpunt mee voor de analyse als het
geobserveerd habitatsubtype en het doelhabitatsubtype tot hetzelfde
habitattype behoren. In andere gevallen wordt het meetpunt niet
weerhouden. Ook meetpunten waarvoor de cirkelplot doelhabitat bevat en
de vierkante plot niet, worden niet weerhouden. Bij sommige meetpunten
ontbreekt de vegetatieopname of structuuropname. Ook deze worden niet
weerhouden.

Voor enkele meetpunten wordt het subtype van een habitat niet
gespecifieerd. We gaan er dan vanuit dat het subtype met het doelhabitat
overeenkomt.

\subsection{Overzicht meetpunten}\label{overzicht-meetpunten}

Tabel \ref{tab:tabAantallen} geeft een overzicht van het huidige aantal
opgemeten meetpunten en de totale steekproefgrootte die we na 12 jaar
willen bereiken.

\begin{table}[!h]

\caption{\label{tab:tabAantallen}Aantal opgemeten meetpunten en totaal aantal gewenste meetpunten na meetcyclus van 12 jaar}
\centering
\begin{tabular}{llrr}
\toprule
HabCode & SBZH & nOpgemeten & nGewenst\\
\midrule
1330\_hpr & Binnen & 16 & 73\\
1330\_hpr & Buiten & 0 & 21\\
6120 & Binnen & 4 & 98\\
6120 & Buiten & 0 & 38\\
6230\_ha & Binnen & 15 & 72\\
\addlinespace
6230\_ha & Buiten & 3 & 17\\
6230\_hmo & Binnen & 20 & 58\\
6230\_hmo & Buiten & 3 & 11\\
6230\_hn & Binnen & 17 & 49\\
6230\_hn & Buiten & 8 & 34\\
\addlinespace
6230\_hnk & Binnen & 1 & 2\\
6230\_hnk & Buiten & 0 & 0\\
6410\_mo & Binnen & 23 & 29\\
6410\_mo & Buiten & 9 & 23\\
6410\_ve & Binnen & 11 & 97\\
\addlinespace
6410\_ve & Buiten & 1 & 25\\
7140\_meso & Binnen & 23 & 87\\
7140\_meso & Buiten & 0 & 13\\
7140\_oli & Binnen & 11 & 73\\
7140\_oli & Buiten & 0 & 1\\
\bottomrule
\end{tabular}
\end{table}

\needspace{50mm}

\section{LSVI-berekening per
meetpunt}\label{lsvi-berekening-per-meetpunt}

\needspace{50mm}

De onderstaande voorwaarden werden bepaald in de cirkelplot en de
waarden ervan worden rechtstreeks ingevoerd in de LSVI-rekenmodule (via
attribuut data\_voorwaarden, zie paragraaf \ref{h:datainvoer}).

\begin{itemize}
\tightlist
\item
  bedekking strooisellaag
\item
  bedekking structuurschade
\item
  bedekking verbossing
\item
  bedekking microreliëf
\item
  microreliëf aanwezig
\item
  bedekking naakte bodem
\end{itemize}

Eén van de voorwaarden van de indicator
`verruiging/vermossing/vergrassing' uit LSVI versie 2.0 van
circum-neutraal overgangsveen (7140\_meso) kan niet berekend worden,
namelijk de voorwaarde `gemiddelde vegetatiehoogte in cm'. De status van
deze indicator evalueren we daarom enkel op basis van de voorwaarde
`bedekking verruiging/vergrassing/vermossing'.

Onderstaande indicatoren konden niet bepaald worden en werden daarom
niet meegerekend bij de bepaling van de status per meetpunt:

\begin{itemize}
\tightlist
\item
  Horizontale structuur voor habitatsubtype 1330\_hpr (LSVI versie 2 en
  LSVI versie 3)
\item
  Horizontale structuur voor habitatsubtype 7140\_oli en 7140\_meso
  (LSVI versie 2)
\end{itemize}

Alle resterende voorwaarden worden via de LSVI-rekenmodule berekend op
basis van de gegevens van de vegetatieopname (die ingevoerd worden via
attribuut data\_soortenKenmerken, zie paragraaf \ref{h:datainvoer}). De
indicatoren `invasieve exoten' en `sleutelsoorten' leiden we af uit de
vegetatieopname in de cirkelplot. De overige indicatoren (die afgeleid
kunnen worden uit een vegetatieopname) leiden we af uit de
vegetatieopname in de vierkante plot. Dit doen we door een
LSVI-berekening uit te voeren voor zowel de cirkelplot als de vierkante
plot en vervolgens de uitkomst van de rekenmodule voor de indicatoren te
combineren.

\section{Uitspraak Vlaanderen en de Vlaams-Atlantische
regio}\label{uitspraak-vlaanderen-en-de-vlaams-atlantische-regio}

Gezien er slechts 1 meetpunt in de Vlaams-Continentale regio (Voeren)
gelegen is, maken we hier geen onderscheid tussen Vlaanderen en de
Vlaams-Atlantische regio. We maken dus een schatting van het
oppervlakteaandeel met een gunstige kwaliteit op basis van alle
meetpunten.

\section{Resultaten}\label{resultaten}

De resultaten worden weggeschreven in de folder
`AnalyseGraslandMoeras\_2018-12-05'.

\chapter{Heidehabitats en soortenrijke
glanshavergraslanden}\label{heidehabitats-en-soortenrijke-glanshavergraslanden}

\section{Data}\label{data-1}

\subsection{Ruwe data}\label{ruwe-data}

De heidehabitats (2310, 2330, 4010, 4030) en het habitattype
soortenrijke glanshavergraslanden (6510) worden opgemeten door het ANB,
die de gegevens invoeren in Fieldmap. De gegevens worden vervolgens
vanuit Fieldmap naar een Access-bestand geëxporteerd en aangeleverd aan
het INBO.

Dit Access-bestand bevat:

\begin{itemize}
\tightlist
\item
  Geobserveerde habitattypen in de meetpunten,
\item
  Vegetatieopname in de vierkante plot,
\item
  Structuurgegevens in de cirkelplot.
\end{itemize}

\subsection{Overzicht meetpunten}\label{overzicht-meetpunten-1}

Tabel \ref{tab:Heide6510tabAantallen} geeft een overzicht van het
huidige aantal opgemeten meetpunten en de totale steekproefgrootte die
we na 12 jaar willen bereiken.

\begin{table}[t]

\caption{\label{tab:Heide6510tabAantallen}Aantal opgemeten meetpunten en totaal aantal gewenste meetpunten na meetcyclus van 12 jaar}
\centering
\begin{tabular}{llrr}
\toprule
HabCode & SBZH & nOpgemeten & nGewenst\\
\midrule
2310 & Binnen & 52 & 169\\
2310 & Buiten & 0 & 7\\
2330\_bu & Binnen & 35 & 160\\
2330\_bu & Buiten & 2 & 19\\
4010 & Binnen & 31 & 168\\
\addlinespace
4010 & Buiten & 0 & 3\\
4030 & Binnen & 35 & 169\\
4030 & Buiten & 2 & 13\\
6510\_hu & Binnen & 26 & 141\\
6510\_hu & Buiten & 4 & 59\\
\addlinespace
6510\_hua & Binnen & 0 & 0\\
6510\_hua & Buiten & 7 & 73\\
6510\_huk & Binnen & 5 & 52\\
6510\_huk & Buiten & 1 & 19\\
6510\_hus & Binnen & 3 & 44\\
\addlinespace
6510\_hus & Buiten & 3 & 13\\
\bottomrule
\end{tabular}
\end{table}

\needspace{200mm}

\section{LSVI-berekening per
meetpunt}\label{lsvi-berekening-per-meetpunt-1}

\subsection{Heide}\label{heide}

Voor de heidehabitats worden de volgende voorwaarden afgeleid uit de
gegevens ingezameld in de cirkelplot en worden de waarden rechtstreeks
ingevoerd in de LSVI-rekenmodule:

\begin{itemize}
\tightlist
\item
  bedekking verbossing,
\item
  bedekking dwergstruiken,
\item
  bedekking korstmosvegetaties,
\item
  climax- of degeneratiestadium aanwezig,
\item
  aantal ouderdomsstadia,
\item
  aantal talrijke ouderdomsstadia,
\item
  aantal ontwikkelingsstadia,
\item
  bedekking moslaag,
\item
  bedekking veenmoslaag,
\item
  bedekking naakte bodem,
\item
  bedekking open vegetaties,
\item
  bedekking open zand.
\end{itemize}

De indicator `mozaïek met 2330' leiden we af uit de Habitatkaart
\citep{DeSaeger2018}. Ook de waarde van deze voorwaarde wordt
rechtstreeks ingevoerd in de LSVI-rekenmodule.

De indicator `horizontale structuur' met voorwaarde `afwisseling
dopheidebulten en natte slenken' van LSVI versie 2.0 van habitattype
4010 kan niet bepaald worden en wordt daarom niet meegerekend in de
verdere analyse.

De overige voorwaarden van de heidehabitats berekent de LSVI-rekenmodule
op basis van de vegetatieopname in de vierkante plot.

\subsection{Soortenrijke glanshavergraslanden
(6510)}\label{soortenrijke-glanshavergraslanden-6510}

Voor habitattype 6510 bepalen we enkel de voorwaarde `bedekking
verbossing' op het niveau van de cirkelplot. De waarde voeren we
rechtstreeks in in de LSVI-rekenmodule. Alle andere voorwaarden worden
berekend op basis van de vegetatieopname in de vierkante plot.

\section{Uitspraak Vlaanderen en de Vlaams-Atlantische
regio}\label{uitspraak-vlaanderen-en-de-vlaams-atlantische-regio-1}

Enkel voor habitat 6510 maken we een onderscheid tussen Vlaanderen en de
Vlaams-Atlantische regio. De opgemeten meetpunten van de heidehabitats
liggen allemaal in de Atlantische regio.

\section{Resultaten}\label{resultaten-1}

De resultaten worden weggeschreven in de folder
`AnalyseHeide6510\_2018-11-13'.

\chapter{Synergie MONEOS: buitendijkse schorren (1330\_da) en
wilgenvloedbossen (91E0\_sf)}\label{h:MONEOS}

\section{Data}\label{data-2}

\subsection{Ruwe data uit INBOVEG}\label{ruwe-data-uit-inboveg-1}

De INBOVEG-databank bevat de volgende gegevens voor elk meetpunt:

\begin{itemize}
\item
  vegetatieopname (lijst van alle aanwezige soorten + bedekking per
  soort),
\item
  bedekking van vegetatielagen.
\end{itemize}

\subsection{Structuurvariabelen}\label{structuurvariabelen}

\subsubsection{Wilgenvloedbossen
(91E0\_sf)}\label{wilgenvloedbossen-91e0_sf}

De structuurvariabelen zitten niet in INBOVEG en werden aangeleverd als
een afzonderlijk Excel-bestand. Voor habitat 91E0\_sf gaat het om de
onderstaande voorwaarden:

\begin{itemize}
\tightlist
\item
  aantal groeiklassen aanwezig,
\item
  groeiklasse 5, 6 of 7 aanwezig,
\item
  aantal exemplaren dik dood hout per ha,
\item
  aandeel dood hout,
\item
  grondvlak sleutelsoorten boom- en struiklaag.
\end{itemize}

Deze voorwaarden worden rechtstreeks op het terrein ingeschat en dus
niet afgeleid uit een dendrometrische opname zoals bij de overige
boshabitats (zie Hoofdstuk \ref{h:Boshabitats}). De waarden worden
rechtstreeks ingevoerd in de LSVI-rekenmodule.

De volgende voorwaarde kon niet bepaald worden en wordt daarom niet
meegerekend in de verdere analyse:

\begin{itemize}
\tightlist
\item
  aandeel overstromende vloeden.
\end{itemize}

De overige voorwaarden worden berekend via de LSVI-rekenmodule op basis
van de vegetatieopname.

\needspace{70mm}

\subsubsection{Buitendijkse schorren
(1330\_da)}\label{buitendijkse-schorren-1330_da}

De structuurvariabelen zitten niet in INBOVEG en werden aangeleverd als
afzonderlijk Excel-bestand. Voor de buitendijkse schorren (1330\_da)
gaat het om de onderstaande voorwaarden:

\begin{itemize}
\tightlist
\item
  zowel lage als hoge schorvegetaties aanwezig,
\item
  aanwezigheid kreken, oeverwallen en kommen,
\item
  bedekking riet,
\item
  schorklifvegetaties aanwezig,
\item
  aanwezigheid schorklif/breuksteenbestorting,
\item
  habitattype lager dan het klif,
\item
  intertidale ruimte ter hoogte van gemiddelde hoogwaterstand (GHW)
  aanwezig.
\end{itemize}

Deze voorwaarden werden bepaald op het niveau van een schor. Dit
betekent dat alle meetpunten die binnen een zelfde schor gelegen zijn,
dezelfde waarde voor deze voorwaarden hebben. De waarden van deze
voorwaarden worden rechtstreeks ingevoerd in de LSVI-rekenmodule.

De volgende variabele kon niet bepaald worden en wordt daarom niet
meegerekend in de verdere analyse:

\begin{itemize}
\tightlist
\item
  structuurvariatie binnen de verschillende zones aanwezig.
\end{itemize}

Alle overige voorwaarden worden berekend via de LSVI-rekenmodule op
basis van de vegetatieopname.

\section{Overzicht meetpunten}\label{overzicht-meetpunten-2}

Tabel \ref{tab:MONEOStabAantallen} geeft een overzicht van het huidige
aantal meetpunten binnen MONEOS en de totale steekproefgrootte die we na
12 jaar willen bereiken. Tabel \ref{tab:MONEOStabAantallenJaar} geeft
een overzicht van de uitgevoerde vegetatieopnames in de meetpunten. Voor
heel wat meetpunten werden zowel in 1995 als in 2013 een vegetatieopname
uitgevoerd. Maar de structuurgegevens werden enkel in 2018 bepaald.
Daarom zullen we enkel gebruik maken van de vegetatiegegevens uit 2013.

\begin{table}[!h]

\caption{\label{tab:MONEOStabAantallen}Aantal opgemeten meetpunten en totaal aantal gewenste meetpunten na meetcyclus van 12 jaar}
\centering
\begin{tabular}{llrr}
\toprule
Habitattype & SBZH & nOpgemeten & nGewenst\\
\midrule
1330\_da & Binnen & 67 & 79\\
1330\_da & Buiten & 0 & 6\\
91E0\_sf & Binnen & 29 & 67\\
91E0\_sf & Buiten & 1 & 11\\
\bottomrule
\end{tabular}
\end{table}

\begin{table}[!h]

\caption{\label{tab:MONEOStabAantallenJaar}Overzicht van het aantal meetpunten waarvoor een vegetatieopname werd uitgevoerd per jaar en het aantal meetpunten met een vegetatieopname in beide jaren (nHerhaling)}
\centering
\begin{tabular}{llrrr}
\toprule
Habitattype & SBZH & Jaar & nOpgemeten & nHerhaling\\
\midrule
1330\_da & Binnen & 1995 & 33 & \\
\cmidrule{1-4}
1330\_da & Binnen & 2013 & 67 & \multirow{-2}{*}{\raggedleft\arraybackslash 33}\\
\cmidrule{1-5}
91E0\_sf & Binnen & 1995 & 27 & \\
\cmidrule{1-4}
91E0\_sf & Binnen & 2013 & 29 & \multirow{-2}{*}{\raggedleft\arraybackslash 27}\\
\cmidrule{1-5}
91E0\_sf & Buiten & 1995 & 1 & \\
\cmidrule{1-4}
91E0\_sf & Buiten & 2013 & 1 & \multirow{-2}{*}{\raggedleft\arraybackslash 1}\\
\bottomrule
\end{tabular}
\end{table}

\section{LSVI-berekening}\label{lsvi-berekening}

De indicator `verruiging' van habitatsubtype 1330\_da is enkel van
toepassing voor zoutwaterschor. Gezien alle meetpunten in brakwaterschor
gelegen zijn, wordt deze indicator niet meegerekend in de analyse.

De indicator `sleutelsoorten' bestaat uit twee voorwaarden: `aantal
sleutelsoorten hoog schor' en `aantal sleutelsoorten laag schor'. De
indicator is gunstig als beide voorwaarden gunstig scoren. De indicator
wordt echter beoordeeld op basis van een vegetatieopname in een plot van
3m x 3m die ofwel in hoog schor ofwel in laag schor gelegen is, waardoor
er niet aan beide voorwaarden kan worden voldaan. Daarom beoordelen we
in deze analyse de indicator `sleutelsoorten' als gunstig wanneer
(minstens) één van beide voorwaarden gunstig scoort.

\section{Uitspraak Vlaanderen en de Vlaams-Atlantische
regio}\label{uitspraak-vlaanderen-en-de-vlaams-atlantische-regio-2}

Voor habitattype 1330 (Atlantische schorren) maken we een schatting van
het aandeel habitat dat gunstig is in Vlaanderen en berekenen we het
bijhorende 95\%-betrouwbaarheidsinterval. Hiervoor gebruiken we de
resultaten voor subtype 1330\_da (buitendijkse schorren) en 1330\_hpr
(zilte graslanden) (zie
\href{mailto:hoofdstuk@ref}{\nolinkurl{hoofdstuk@ref}}(h:GrasMoeras)).
Voor het habitatsubtype 1330\_da gebruiken we de resultaten gebaseerd op
de opnames die dateren van 2013.

De resultaten voor 91E0\_sf worden geïntegreerd in de analyse van de
boshabitats (zie paragraaf \ref{h:Boshabitats}).

\section{resultaten}\label{resultaten-2}

De resultaten zijn terug te vinden in de folder
`AnalyseMONEOS\_2019-01-14'.

\chapter{Boshabitats}\label{h:Boshabitats}

Voor de boshabitats maken we gebruik van meetpunten uit de
Bosinventarisatie \citep{Wouters2008c} en van bijkomende meetpunten uit
het meetnet habitatkwaliteit \citep{Westra2014}.

\section{Data}\label{data-3}

\subsection{Ruwe Data}\label{ruwe-data-1}

\subsubsection{Bosinventarisatie}\label{bosinventarisatie}

Een groot deel van de gegevens voor de boshabitats wordt via de
Bosinventarisatie ingezameld. De gegevens van de tweede
Bosinventarisatie (2009 - 2018) worden door ANB ingevoerd in Fieldmap.
De gegevens worden vervolgens vanuit Fieldmap naar een Access-bestand
geëxporteerd en aangeleverd aan het INBO. Ook de gegevens van de eerste
Bosinventarisatie (1997 - 1999) zitten in een Access-bestand.

Deze Access-bestanden bevatten:

\begin{itemize}
\tightlist
\item
  een vegetatieopname in de vierkante plot,
\item
  dendrometrische gegevens in de cirkelplot,
\item
  een bestandsbeschrijving in de cirkelplot.
\end{itemize}

Het habitattype waarin elk meetpunt gelegen is, werd niet op het terrein
bepaald (dit zal in de toekomst wel gebeuren). Daarom maken we (1) een
overlay tussen de meetpunten van de Bosinventarisatie en de Habitatkaart
\citep{DeSaeger2018} om te bepalen welke punten in boshabitat vallen,
waarna we vervolgens (2) de punten selecteren die in een polygoon van de
Habitatkaart vallen met meer dan 50\% van een bepaald boshabitattype.

\subsubsection{Bijkomende meetpunten van
Habitatkwaliteitsmeetnet}\label{bijkomende-meetpunten-van-habitatkwaliteitsmeetnet}

De gegevens van het Habitatkwaliteitsmeetnet \citep{Westra2014} worden
eveneens ingezameld door ANB en aangeleverd aan INBO onder de vorm van
een Access-bestand. De meetpunten worden op dezelfde manier opgemeten
als in de Bosinventarisatie. Bijkomend wordt ook het habitattype per
meetpunt bepaald.

\subsection{Overzicht meetpunten}\label{overzicht-meetpunten-3}

Tabel \ref{tab:BostabAantallen} geeft een overzicht van het aantal
opgemeten meetpunten in de tweede Bosinventarisatie en het
Habitatkwaliteitsmeetnet, en de totale steekproefgrootte die we na 12
jaar willen bereiken. Deze dataset gebruiken we om de toestand te
schatten: het huidige aandeel habitat in een gunstige staat.

\begin{table}[t]

\caption{\label{tab:BostabAantallen}Aantal opgemeten meetpunten in de 2de Bosinventarisatie (VBI2) en in het meetnet habitatkwaliteit (MHK), en totaal aantal gewenste meetpunten na de meetcyclus van 12 jaar}
\centering
\begin{tabular}{llrrrr}
\toprule
HabCode & SBZH & nOpgemeten VBI2 & nOpgemeten MHK & nOpgemeten totaal & nGewenst\\
\midrule
9110 & Binnen & 2 & 0 & 2 & 0\\
9110 & Buiten & 1 & 0 & 1 & 0\\
9120 & Binnen & 141 & 1 & 142 & 160\\
9120 & Buiten & 121 & 0 & 121 & 37\\
9130\_end & Binnen & 27 & 2 & 29 & 165\\
\addlinespace
9130\_end & Buiten & 8 & 1 & 9 & 31\\
9160 & Binnen & 24 & 3 & 27 & 168\\
9160 & Buiten & 17 & 2 & 19 & 36\\
9190 & Binnen & 27 & 3 & 30 & 168\\
9190 & Buiten & 27 & 1 & 28 & 47\\
\addlinespace
91E0\_va & Binnen & 30 & 0 & 30 & 42\\
91E0\_va & Buiten & 21 & 0 & 21 & 48\\
91E0\_vc & Binnen & 1 & 14 & 15 & 46\\
91E0\_vc & Buiten & 3 & 7 & 10 & 33\\
91E0\_vm & Binnen & 27 & 1 & 28 & 50\\
\addlinespace
91E0\_vm & Buiten & 8 & 1 & 9 & 30\\
91E0\_vn & Binnen & 7 & 0 & 7 & 46\\
91E0\_vn & Buiten & 32 & 0 & 32 & 46\\
91E0\_vo & Binnen & 9 & 6 & 15 & 58\\
91E0\_vo & Buiten & 2 & 1 & 3 & 21\\
\bottomrule
\end{tabular}
\end{table}

\needspace{50mm}

Voor het schatten van veranderingen in habitatkwaliteit gebruiken we
meetpunten die zowel in de eerste als in de tweede Bosinventarisatie
werden opgemeten. In de eerste Bosinventarisatie werden meetpunten die
op een bosrand liggen, verplaatst naar homogeen bos. Tijdens de tweede
Bosinventarisatie werden deze meetpunten echter terug naar de
oorspronkelijke locatie verplaatst omdat dit een meer representatief
beeld geeft van het bos in Vlaanderen \citep{Wouters2008c}. We maken
geen gebruik van deze verplaatste punten om de veranderingen in
habitatkwaliteit te analyseren.

Tabel \ref{tab:BosTrends} geeft een overzicht van het aantal meetpunten
die:

\begin{itemize}
\tightlist
\item
  zowel in de eerste als in de tweede Bosinventarisatie werden opgemeten
  EN
\item
  die niet werden verplaatst in de eerste Bosinventarisatie EN
\item
  waarvoor zowel een dendrometrische opname als een vegetatieopname werd
  uitgevoerd (in de eerste Bosinventarisatie werd slechts in de helft
  van de meetpunten een vegetatieopname uitgevoerd).
\end{itemize}

Deze dataset gebruiken we om veranderingen te schatten in het aandeel
habitat met gunstige kwaliteit tussen de periode 1997 - 1999 (eerste
Bosinventarisatie) en de periode 2009 - 2018 (tweede Bosinventarisatie).

\begin{table}[t]

\caption{\label{tab:BosTrends}Aantal meetpunten die zowel in de eerste als in de tweede Bosinventarisatie werden opgemeten}
\centering
\begin{tabular}{llr}
\toprule
HabCode & SBZH & nOpgemeten VBI\\
\midrule
9110 & Binnen & 1\\
9120 & Binnen & 63\\
9120 & Buiten & 54\\
9130\_end & Binnen & 11\\
9130\_end & Buiten & 3\\
\addlinespace
9160 & Binnen & 11\\
9160 & Buiten & 8\\
9190 & Binnen & 9\\
9190 & Buiten & 8\\
91E0\_va & Binnen & 11\\
\addlinespace
91E0\_va & Buiten & 8\\
91E0\_vm & Binnen & 11\\
91E0\_vm & Buiten & 4\\
91E0\_vn & Binnen & 1\\
91E0\_vn & Buiten & 9\\
\addlinespace
91E0\_vo & Binnen & 2\\
91E0\_vo & Buiten & 1\\
\bottomrule
\end{tabular}
\end{table}

\needspace{60mm}

\section{LSVI-berekening per
meetpunt}\label{lsvi-berekening-per-meetpunt-2}

Voor de volgende voorwaarden worden de waarden rechtstreeks ingevoerd in
de LSVI-rekenmodule:

\begin{itemize}
\tightlist
\item
  aandeel dood hout,
\item
  hoeveelheid dik dood hout,
\item
  bosconstantie,
\item
  minimum structuurareaal (MSA).
\end{itemize}

\needspace{60mm} De \emph{bosconstantie} wordt afgeleid uit de
bosleeftijdskaart en de bestandsleeftijd opgemeten op het terrein:

\begin{itemize}
\tightlist
\item
  bosconstantie \textgreater{}= 100 jaar als het meetpunt tot de klasse
  `voor 1775' of `tussen 1775 en 1850' behoort OF als de
  bestandsleeftijd \textgreater{} 100 jaar;
\item
  bosconstantie \textgreater{}= 75 jaar als het meetpunt tot de klasse
  `tussen 1850 en +-1930' behoort OF als de bestandsleeftijd
  \textgreater{} 80 jaar;
\item
  bosconstantie \textgreater{}= 30 jaar als het meetpunt tot de klasse
  `na +-1930' behoort EN als de bestandsleeftijd \textgreater{} 40 jaar
  heeft;
\item
  bosconstantie \textless{} 30 jaar in alle andere gevallen.
\end{itemize}

\emph{Aandeel dood hout} en \emph{hoeveelheid dik dood hout} worden
berekend op basis van de dendrometrische gegevens. Beide variabelen
worden steeds voor de volledige plot berekend, ook al bestaat het
meetpunt slechts gedeeltelijk uit doelhabitat. Voor de schatting van de
toestand op basis van de tweede Bosinventarisatie en het Meetnet
Habitatkwaliteit, gebruiken we gegevens van liggend en staand dood hout
om beide voorwaarden te berekenen. In de eerste Bosinventarisatie werd
echter enkel staand dood hout opgemeten. Voor de schatting van
veranderingen tussen beide inventarisaties gebruiken we dus enkel
gegevens van staand dood hout, zodat de resultaten voor beide periodes
vergelijkbaar zijn. We gebruiken dan ook een aangepaste referentiewaarde
voor de indicator `aandeel dood hout', meer bepaald 2 m²/ha i.p.v. 4
m²/ha.

Het is niet aangewezen om de indicator `hoeveelheid dik dood hout' te
beoordelen op meetpuntniveau. Dik dood hout is immers dermate zeldzaam,
dat dit op niveau van een boscomplex zou moeten beoordeeld worden.
Daarom rekenen we de beoordeling van deze indicator niet mee bij het
bepalen van de status van de habitatvlek. Wel schatten we het gemiddeld
aantal exemplaren dik dood hout per hectare voor geheel Vlaanderen. Door
dit voor beide periodes van de Bosinventarisatie te doen, geeft dit een
idee hoe de indicator evolueert in de tijd.

\needspace{50mm} De overige voorwaarden van de boshabitats worden via de
LSVI-rekenmodule berekend op basis van de volgende gegevens:

\begin{itemize}
\tightlist
\item
  de bedekking van de soorten in de vegetatieplot,
\item
  de bedekking van de vegetatielagen in de vegetatieplot,
\item
  de aanwezige groeiklassen,
\item
  het grondvlak per boomsoort.
\end{itemize}

De aanwezige groeiklassen worden afgeleid uit de vegetatiegegevens
(groeiklasse 2) en de dendrometrische gegevens (groeiklassen 3 tot 7).
Groeiklasse 1 (= open ruimte in bos) kan niet afgeleid worden uit de
meetgegevens en ontbreekt dus steeds. Wanneer een meetpunt slechts
gedeeltelijk uit doelhabitat bestaat, tellen we toch alle aanwezige
groeiklasse binnen het volledige meetpunt mee. Een dikke boom uit
groeiklasse 7 die niet in het doelhabitat ligt maar wel binnen het
meetpunt valt (de cirkelplot met straal van 18 meter) zal dus toch
meegerekend worden.

Het grondvlak per boomsoort leiden we af uit de dendrometrische
gegevens. Als een meetpunt slechts gedeeltelijk uit doelhabitat bestaat,
zullen we hier enkel de bomen meerekenen die gelegen zijn binnen het
deel van de plot met doelhabitat. Op basis deze gegevens wordt immers de
voorwaarde `grondvlakaandeel van de sleutelsoorten in de boomlaag'
bepaald. Een uitzondering hierop is habitatsubtype 91E0\_vc, dat vaak
slechts over een kleine oppervlakte voorkomt en waarvoor ook de bomen in
de omliggende habitatvlekken (maar binnen het meetpunt) meegenomen
worden voor het bepalen van het aandeel sleutelsoorten. Ten slotte
rekenen we, zoals aangegeven in \citet{TJollyn2009} en
\citet{Oosterlynck2018}, bij habitattype 91E0 het grondvlak van
populieren niet mee als de bedekking van de boomlaag zonder populier
groter is dan 70 \%. De bedekking van de boomlaag zonder populier leiden
we af uit de vegetatieopname.

De voorwaarde `schaalgrootte ingrepen (ha)', die onderdeel uitmaakt van
de indicator `horizontale structuur - natuurlijke mozaiekstructuur', kan
niet worden afgeleid uit de beschikbare gegevens en en wordt daarom niet
meegerekend bij de evaluatie van de LSVI.

\section{Uitspraak Vlaanderen en de Vlaams-Atlantische
regio}\label{uitspraak-vlaanderen-en-de-vlaams-atlantische-regio-3}

We maken enerzijds een schatting van de huidige toestand op basis van de
meetpunten van de tweede Bosinventarisatie en het Meetnet
Habitatkwaliteit.

Daarnaast maken we ook een schatting voor beide periodes op basis van de
meetpunten die zowel in de eerste Bosinventarisatie als in de tweede
Bosinventarisatie werden opgemeten. Op basis van deze schattingen kunnen
veranderingen gedetecteerd worden.

Voor habitattype 91E0 maken we ook gebruik van de resultaten van
habitatsubtype 91E0\_sf die werden bekomen op basis van de
MONEOS-gegevens (zie Hoofdstuk \ref{h:MONEOS}). We gebruiken daarvoor
enkel de gegevens die in 2013 werden ingezameld.

Gezien er enkele meetpunten met boshabitat in de Continentale regio
(Voeren) gelegen zijn, maken we schatting voor Vlaanderen en voor de
Vlaams-Atlantische regio.

\section{Resultaten}\label{resultaten-3}

De resultaten worden weggeschreven in de folder
`AnalyseBoshabitats\_2019-01-15'.

\chapter{Synergie PINK:
kustduinhabitats}\label{synergie-pink-kustduinhabitats}

\section{Embryonale duinen (2110)}\label{embryonale-duinen-2110}

Voor het habitattype embryonale duinen (2110) werd geen meetnet
ontwikkeld omdat het om een zeer dynamisch habitattype gaat
\citep{Westra2014}. In de plaats daarvan wordt de habitatkwaliteit
ingeschat via kartering. De kartering werd uitgevoerd in het kader van
PINK \citep{Provoost2015}.

\subsection{Data}\label{data-4}

De karteergegevens werden aangeleverd onder de vorm van een
Excel-bestand met per polygoon een aantal structuur- en
vegetatiegegevens waaruit het merendeel van de verschillende
indicatoren/voorwaarden afgeleid kunnen worden.

\subsection{LSVI-berekening}\label{lsvi-berekening-1}

De waarden van alle voorwaarden worden rechtstreeks ingevoerd in de
LSVI-rekenmodule. Voor de voorwaarden `sleutelsoorten structuurvormend'
en `bedekking Biestarwegras' waren er echter geen gegevens beschikbaar.
Beide voorwaarden beschouwen we steeds als gunstig voor alle polygonen
op basis van expertinschatting.

Op basis van de LSVI-rekenmodule berekenen we voor elke polygoon van de
kartering de habitatkwaliteit. De verhouding van de oppervlakte
polygonen in een gunstige staat t.o.v. de totale gekarteerde oppervlakte
geeft dan het aandeel habitat in een gunstige staat.

\section{Overige kustduinhabitats}\label{overige-kustduinhabitats}

Voor de habitattypen vastgelegde duinen (2130), duindoornstruwelen
(2160), kruipwilgstruwelen (2170) en vochtige duinvalleien (2190) maken
we gebruik van vegetatieopnames in vierkante proefvlakken van 3 m x 3 m
die uitgevoerd werden in het kader van PINK.

\needspace{50mm}

\subsection{Data}\label{data-5}

\subsubsection{Ruwe data uit INBOVEG}\label{ruwe-data-uit-inboveg-2}

De INBOVEG-databank bevat de volgende gegevens voor elke meetpunt:

\begin{itemize}
\tightlist
\item
  vegetatieopname (lijst van alle aanwezige soorten + bedekking per
  soort),
\item
  bedekking van vegetatielagen.
\end{itemize}

\subsubsection{Geobserveerd habitattype}\label{geobserveerd-habitattype}

De shapefile `PQ\_Duinen\_20180420' bevat de ligging van de meetpunten
en het geobserveerde habitattype. Elk meetpunt heeft een ID waarmee de
link naar de opname in INBOVEG kan worden gemaakt.

\subsubsection{Gegevens verbossing voor
duindoornstruwelen}\label{gegevens-verbossing-voor-duindoornstruwelen}

Voor het beoordelen van de voorwaarde `verbossing' bij het habitattype
duindoornstruwelen (2160) maken we gebruik van karteergegevens. De
inschatting van verbossing op basis van een kartering is nauwkeuriger
dan deze afgeleid uit een vegetatieopname in een proefvlak van 3m x 3m.

\subsubsection{Overzicht meetpunten}\label{overzicht-meetpunten-4}

Tabel \ref{tab:PINKtabAantallenToestand} geeft een overzicht van het
aantal meetpunten binnen PINK waarvoor een opname is gebeurd in de
periode 2010-2017 en de totale steekproefgrootte die we na 12 jaar
willen bereiken. Daarbij moet opgemerkt worden dat alle meetpunten
binnen ANB-domeinen gelegen zijn en bepaalde beheervormen relatief vaker
bemonsterd zijn dan anderen. De meetpunten vormen dus geen
representatieve steekproef voor Vlaanderen.

\begin{table}[!h]

\caption{\label{tab:PINKtabAantallenToestand}Aantal meetpunten waarvoor een opname gebeurd is in de periode 2010-2017 en totaal aantal gewenste meetpunten na meetcyclus van 12 jaar}
\centering
\begin{tabular}{llrr}
\toprule
Habitattype & SBZH & nOpgemeten & nGewenst\\
\midrule
2130\_had & Binnen & 29 & 73\\
2130\_had & Buiten & 0 & 2\\
2130\_hd & Binnen & 159 & 135\\
2130\_hd & Buiten & 0 & 7\\
2160 & Binnen & 41 & 165\\
\addlinespace
2160 & Buiten & 0 & 6\\
2170 & Binnen & 11 & 126\\
2170 & Buiten & 0 & 1\\
2190 & Binnen & 34 & 50\\
2190 & Buiten & 0 & 1\\
\addlinespace
2190\_mp & Binnen & 44 & 77\\
2190\_mp & Buiten & 0 & 0\\
\bottomrule
\end{tabular}
\end{table}

\subsection{LSVI-berekening per
meetpunt}\label{lsvi-berekening-per-meetpunt-3}

Er zijn geen gegevens beschikbaar om de indicator `horizontale
structuur' van het habitattype duindoornstruwelen (2160) te beoordelen.
Deze indicator is gebaseerd op de aanwezigheid van open plekken en het
aantal struweeltypes. Ook voor de indicator `ouderdomsstructuur
Duindoorn' van habitattype 2160 zijn er geen gegevens beschikbaar om de
beoordeling te kunnen uitvoeren. Beide indicatoren worden daarom niet
meegenomen in de verdere analyse.

Verder geven we enkel voor de voorwaarde `verbossing' bij habitattype
2160 de waarden rechtstreeks in in de LSVI-rekenmodule. De waarden voor
de overige voorwaarden worden berekend door de LSVI-rekenmodule op basis
van de gegevens van de vegetatieopname.

Voor het habitattype 2190 is er maar één beoordelingstabel beschikbaar
in \citet{TJollyn2009} en \citet{Oosterlynck2018}, namelijk deze voor
het subtype `duinpannen met kalkminnende vegetaties' (2190\_mp).
Meetpunten met habitattype 2190 die niet tot het subtype 2190\_mp
behoren worden daarom beoordeeld op basis van de LSVI-beoordelingstabel
van 2190\_mp.

\subsection{Uitspraak Vlaanderen en de Vlaams-Atlantische
regio}\label{uitspraak-vlaanderen-en-de-vlaams-atlantische-regio-4}

De kustduinen liggen allemaal in de Atlantische regio. Bijgevolg gelden
de resultaten voor Vlaanderen evenzeer voor de Vlaams-Atlantische regio.

\section{Resultaten}\label{resultaten-4}

De resultaten worden weggeschreven in de folder
`AnalysePINK\_2019-01-14':

\chapter{Stilstaande wateren}\label{h:plassen}

\section{Data}\label{data-6}

\subsection{Ruwe data uit INBOVEG}\label{ruwe-data-uit-inboveg-3}

Het veldwerk voor een LSVI-opname van een meetpunt in stilstaande
wateren bestaat uit een vegetatieopname van een volledige plas. De
werkwijze hiervoor is beschreven door \citet{Oosterlynck2018} en
\citet{Westra2014}. Zowel de gegevens voor de toepassing van LSVI-versie
2 \citep{TJollyn2009} als voor versie 3 \citep{Oosterlynck2018} werden
tijdens het veldwerk genoteerd.

De ruwe data gebruikt voor de LSVI-bepalingen in functie van de Natura
2000-rapportage 2013-2018 zijn opgenomen in INBOVEG (survey 196:
HT31xx\_LSVIPlassen). Enkel waarnemingen met een zekere
habitatclassificatie werden opgenomen in de analyse (Classification:
NotSure = 0). De oudste opname dateert van 1-09-2014; de recentste
opname is van 19-10-2018. De INBOVEG databank bevat, naast algemene
informatie (datum, waarnemer, locatie, locatiecode), de volgende
gegevens voor elk meetpunt:

\begin{itemize}
\tightlist
\item
  habitattype,
\item
  beperkte vegetatieopname: 6-delige Tansley-schatting per sleutelsoort
  en per verstoringsindicator,
\item
  procentuele bedekking van verstoringsindicatoren,
\item
  structuurvariabelen die nodig zijn voor de LSVI-berekening
  (afhankelijk van het habitat(sub)type).
\end{itemize}

Via een MS Access-frontend (iv-query-prd\_versie2\_LSVIMeren.mdb) werd
INBOVEG bevraagd; deze gegevens werden via een koppeling ingeladen in
excelbestand. In dit excelbestand zijn ook tabellen opgenomen voor de
omzetting van klassen en tekst zodat de data omgevormd kan worden tot
een geschikt formaat voor de LSVI-rekenmodule.

\subsection{Ruwe data uit LIMS
databank}\label{ruwe-data-uit-lims-databank}

Voor de LSVI-bepaling van habitattype 3160 (versie 2) zijn metingen van
het elektrisch geleidingsvermogen nodig. Dit werd gemeten tijdens de
staalname in het kader van de projecten `Platform Passende Beoordeling,
partim abiotiek oppervlaktewater' (INBOPRJ-10798) en `Meetnet abiotiek
Natura 2000 habitattypen: oppervlaktewater' (INBOPRJ-9430). De
meetresultaten zijn opgeslagen in de LIMS-databank. De metingen werden
meermaals per staalnamejaar uitgevoerd; het gemiddelde ervan werd
berekend en gebruikt in de analyse.

\subsection{Overzicht meetpunten}\label{overzicht-meetpunten-5}

Tabel \ref{tab:tabAantallenPlassen} geeft een overzicht van het huidige
aantal opgemeten meetpunten en de totale steekproefgrootte. Habitattype
3110 en 3140 worden integraal bemonsterd. Dit betekent dat alle gekende
waterlichamen die tot deze habitattypen behoren, worden opgemeten. Het
gewenste aantal meetpunten voor deze habitattypes zijn gebaseerd op de
meest recente versie van de Habitatkaart \citep{DeSaeger2018} en kunnen
daarom verschillen van deze vermeld in \citep{Westra2014}.

\begin{table}[!h]

\caption{\label{tab:tabAantallenPlassen}Aantal opgemeten meetpunten en totaal aantal gewenste meetpunten}
\centering
\begin{tabular}{rlrr}
\toprule
HabCode & SBZH & nOpgemeten & nGewenst\\
\midrule
3110 & Binnen & 2 & 5\\
3130 & Binnen & 71 & 109\\
3130 & Buiten & 7 & 29\\
3140 & Binnen & 16 & 18\\
3140 & Buiten & 3 & 14\\
\addlinespace
3150 & Binnen & 24 & 53\\
3150 & Buiten & 5 & 24\\
3160 & Binnen & 23 & 31\\
3160 & Buiten & 2 & 5\\
\bottomrule
\end{tabular}
\end{table}

\section{LSVI-berekening per
meetpunt}\label{lsvi-berekening-per-meetpunt-4}

Voor het merendeel van de voorwaarden worden de waarden rechtstreeks
ingevoerd in de LSVI-rekenmodule. Enkel de voorwaarden met betrekking
tot sleutelsoorten en de voorwaarde `bedekking helofyten' (habitattype
3130\_aom) worden berekend door de LSVI-rekenmodule op basis van de
vegetatiegegevens.

De indicator `doorzicht' wordt beoordeeld op basis van de secchidiepte.
Wanneer de bodem zichtbaar is, zullen we deze indicator steeds als
gunstig beschouwen onafhankelijk van de diepte van het waterlichaam. Dit
doen we in de praktijk door een secchidiepte van 4 meter in te geven in
de rekenmodule. Deze waarde is groter dan de referentiewaarde, wat dus
resulteert in een gunstige beoordeling. Ook bij droogval beschouwen we
de indicator gunstig en gaan we op dezelfde manier te werk.

\section{Uitspraak Vlaanderen en de Vlaams-Atlantische
regio}\label{uitspraak-vlaanderen-en-de-vlaams-atlantische-regio-5}

Gezien er geen meetpunten in de Vlaams-Continentale regio gelegen zijn,
is de uitspraak voor Vlaanderen dezelfde als deze voor de
Vlaams-Atlantische regio. De werkwijze om tot een uitspraak voor
Vlaanderen te komen, is verschillend voor een steekproefbenadering en
een integrale bemonstering.

\subsection{Steekproefbenadering (3130, 3150,
3160)}\label{steekproefbenadering-3130-3150-3160}

Voor de habitattypen 3130, 3150 en 3160 maken we gebruik van een
steekproef om tot een uitspraak te komen voor Vlaanderen. Net zoals voor
de terrestrische habitats, zullen we op basis van de steekproef een
schatting maken van het oppervlakteaandeel habitat in een gunstige staat
met bijhorende betrouwbaarheidsintervallen.

Om tot een representatieve uitspraak te komen, moeten we ook hier
gebruik maken van meetpuntgewichten (zie paragraaf
\ref{h:meetpuntgewichten}). De strata worden bij de stilstaande wateren
gevormd door een combinatie van de ligging t.o.v. SBZH (binnen SBZH en
buiten SBZH) en de oppervlakteklassen van de waterlichamen (\textless{}
1 ha; 1 - 5 ha; 5 - 50 ha) \citep{Westra2014}. Het stratumgewicht is
omgekeerd evenredig met het oppervlakteaandeel van de stilstaande
wateren dat bemonsterd is binnen elk stratum. Het gewicht van het
waterlichaam is evenredig met de oppervlakte van het waterlichaam. Dus
niet de oppervlakte aan habitat, maar de oppervlakte van het volledige
waterlichaam wordt gebruikt voor de weging. Het meetpuntgewicht is dan
het product van het stratumgewicht en het gewicht van het waterlichaam.

\subsection{Integrale bemonstering (3110, 3140 en plassen \textgreater{}
50 ha)}\label{integrale-bemonstering-3110-3140-en-plassen-50-ha}

Voor habitattype 3110 en 3140 en voor de stilstaande wateren die habitat
bevatten en een oppervlakte hebben groter dan 50 hectare, wordt er een
integrale bemonstering uitgevoerd van alle waterlichamen. Omdat de
eerste monitoringscyclus nog niet volledig is afgewerkt, zijn nog niet
alle veldgegevens beschikbaar. Enkel van de opgemeten plassen kan de
status worden berekend en wordt de oppervlakte meegerekend in het
oppervlakteaandeel gunstig of ongunstig. De oppervlakte van het
waterlichaamsen die nog niet zijn gemeten wordt bij de categorie
`onbekend' gerekend (`area where condition is not known').

\subsection{Combinatie steekproef en integrale
bemonstering}\label{combinatie-steekproef-en-integrale-bemonstering}

Voor habitattype 3130 en 3150 bestaat het meetnet uit een steekproef van
de waterlichamen met een oppervlakte kleiner dan 50 hectare en een
integrale bemonstering van de waterlichamen groter dan 50 hectare. Om
een gecombineerde schatting te bekomen, wordt eerst de oppervlakte
habitat met een gunstige staat geschat voor de waterlichamen kleiner dan
50 hectare met bijhorende betrouwbaarheidsintervallen. Vervolgens wordt
de oppervlakte van de plassen groter dan 50 hectare met een gunstige
kwaliteit hierbij opgeteld.

\section{Resultaten}\label{resultaten-5}

De resultaten zijn terug te vinden in de folder
`AnalyseMeren\_2018-11-06'.

\chapter{Stromende wateren (3260)}\label{h:rivieren}

\section{Data}\label{data-7}

\subsection{Ruwe data uit macrofytendatabank
waterlopen}\label{ruwe-data-uit-macrofytendatabank-waterlopen}

Van de `macrofytendatabank waterlopen' (versie 1.4; 31/07/2018) werden
de volgende opnames gebruikt:

\begin{itemize}
\item
  VMM-opnames: vegetatieopnames van VMM verzameld voor de rapportage van
  de Europese Kaderrichtlijn Water (opnames van 11 juni 2013 - 28
  september 2017)
\item
  INBO-opnames: vegetatieopnames van INBO verzameld voor de
  habitatkwaliteitsmonitoring van habitattype 3260 (opnames van 18 juni
  2015 - 5 september 2017; eveneens beschikbaar via INBOVEG: survey
  HT3260)
\end{itemize}

Van beide datasets werden enkel de opnames gebruikt die het habitattype
bevatten volgens de habitatdefinitie en soortenlijst vermeld door
\citep{Leyssen2018}.

De databank bevat, naast algemene informatie (datum, waarnemer, locatie,
locatiecode, meetnet), de volgende gegevens voor elk meetpunt:

\begin{itemize}
\tightlist
\item
  vegetatieopname van 100m-traject met een 7-delige Tansley-schaal,
\item
  procentuele bedekking van verstoringsindicatoren (helofyten,
  eutrofiëringsindicatoren en invasieve exoten),
\item
  structuurvariabelen die nodig zijn voor de LSVI-berekening
  (oppervlakte grootste vegetatievlek en aantal groeivormen).
\end{itemize}

\subsection{Overzicht meetpunten}\label{overzicht-meetpunten-6}

Tabel \ref{tab:tabelMeetpuntenRivier} geeft een overzicht van het
huidige aantal opgemeten meetpunten en de totale steekproefgrootte na 12
jaar.

\begin{table}[!h]

\caption{\label{tab:tabelMeetpuntenRivier}Aantal opgemeten meetpunten en totaal aantal gewenste meetpunten}
\centering
\begin{tabular}{rlrr}
\toprule
HabCode & SBZH & nOpgemeten & nGewenst\\
\midrule
3260 & Binnen & 34 & 150\\
3260 & Buiten & 12 & 26\\
\bottomrule
\end{tabular}
\end{table}

\section{LSVI-berekening per
meetpunt}\label{lsvi-berekening-per-meetpunt-5}

Voor het merendeel van de voorwaarden worden de waarden rechtstreeks
ingevoerd in de LSVI-rekenmodule. De voorwaarden met betrekking tot
sleutelsoorten worden echter berekend door de LSVI-rekenmodule op basis
van de vegetatiegegevens. Bij de VMM-opnames werd het percentage
invasieve exoten niet genoteerd tijdens het veldbezoek. Voor deze
opnames wordt de voorwaarde berekend door de LSVI-rekenmodule op basis
van de vegetatieopname. Voor de VMM-opnames ontbreekt ook de voorwaarde
`grootste vegetatievlek in m²'. Deze voorwaarde kan niet afgeleid worden
uit de vegetatieopname, wat dus resulteert in een ontbrekende waarde.

\section{Uitspraak Vlaanderen en de Vlaams-Atlantische
regio}\label{uitspraak-vlaanderen-en-de-vlaams-atlantische-regio-6}

We maken opnieuw gebruik van meetpuntgewichten om tot een
representatieve uitspraak te komen voor Vlaanderen en de
Vlaams-Atlantische regio. Het habitatkwaliteitsmeetnet bevat immers
relatief meer meetpunten binnen SBZH dan erbuiten. Daarnaast liggen de
VMM-meetpunten allen in een stroomgebieden met een oppervlakte groter
dan 10 km², terwijl habitattype 3260 ook in stroomgebieden met een
oppervlakte kleiner dan 10 km² voorkomt. Ook hiervoor moet dus
gecorrigeerd worden via de meetpuntgewichten.

De strata bestaan uit de combinatie van de ligging t.o.v. SBZH (binnen
en buiten) en de oppervlakte van het stroomgebied (\textless{} 10 km² en
\textgreater{} 10 km²). Het meetpuntgewicht is omgekeerd evenredig met
het aandeel waterlooptrajecten dat bemonsterd is binnen elk stratum.

\section{Resultaten}\label{resultaten-6}

De resultaten worden weggeschreven in de folder
`AnalyseWaterlopen\_2018-11-06'

\cleardoublepage
\bibliographystyle{inbo}
\bibliography{RapportageHR.bib}
\addcontentsline{toc}{chapter}{\bibname}
\appendix

\chapter{Bijlage: input-bestanden voor LSVI-rekenmodule en
resultaten}\label{bijlage-input-bestanden-voor-lsvi-rekenmodule-en-resultaten}

Als bijlage geven we de input-bestanden voor LSVI-rekenmodule en de
bestanden met de resultaten van de analyse mee. Deze bestanden worden
gegroepeerd in de volgende folders:

\begin{itemize}
\tightlist
\item
  AnalyseGraslandMoeras\_2018-12-05
\item
  AnalyseHeide6510\_2018-11-13
\item
  AnalyseMONEOS\_2019-01-14
\item
  AnalyseBoshabitats\_2019-01-15
\item
  AnalysePINK\_2019-01-14
\item
  AnalyseMeren\_2018-11-06
\item
  AnalyseWaterlopen\_2018-11-06
\end{itemize}


\end{document}
